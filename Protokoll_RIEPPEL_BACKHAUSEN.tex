\documentclass[a4paper,12pt]{scrreprt}
\usepackage[T1]{fontenc}
\usepackage[utf8]{inputenc}
\usepackage[ngerman]{babel}
\usepackage[table]{xcolor}% http://ctan.org/pkg/xcolor
\usepackage{tabu}
\usepackage{graphicx}
\usepackage{lmodern}

\begin{document}


%\titlehead{Kopf} %Optionale Kopfzeile
\author{Alexander Rieppel \and Dominik Backhausen} %Zwei Autoren
\title{ Replikation } %Titel/Thema
\subject{VSDB} %Fach
\subtitle{ Protokoll } %Genaueres Thema, Optional
\date{\today} %Datum
\publishers{5AHITT} %Klasse

\maketitle
\tableofcontents


\chapter{Aufgabenstellung}
	Eine Handelsgesellschaft, die mehrere Filialen hat, betreibt eine Online-Plattform für den Verkauf der Produkte. Der Webshop wird mit Hilfe einer Datenbank betrieben. Bei dem Verkauf der Produkte werden Rechnungen in Form von PDF-Dokumenten erzeugt.\\\\
	Aufgabenstellung:\\
	Die Daten (Datenbank, Rechnungen) sollen stets auf die Filialrechner repliziert werden, damit die Sachbearbeiter vor Ort diese einsehen und bearbeiten können.\\\\
	\begin{itemize}
	\item Entwickle ein vereinfachtes Datenbankmodell für den Webshop
	\item Wähle ein Konsistenzprotokoll Deiner Wahl (siehe Theorie bzw. Tanenbaum)
	\item Implementiere einen Replikationsmanager in Java (JDBC, Sockets, o.ä. ...) für Datenbank und Rechnungen
	\item alle Transaktionen im Zuge der Replikation sollen protokolliert werden (zum Beispiel mit Log4J)
	\end{itemize}
	Beispiel fuer Log-Eintrag:\\
	Replikation Rechnungen München -> Berlin OKAY\\
	Replikation DB MÜnchen -> Berlin FEHLGESCHLAGEN\\\\
	Problemstellungen:
	\begin{itemize}
	\item Wie oft wird repliziert?
	\item Wie erfolgt der Aufruf des Replikationsmanager bzw. läuft der Replikationsmanager stets im Hintergrund?
	\item Was passiert im Fehlerfall?
	\item Welche Probleme können auftreten? (Dateien mit gleichen Namen, Dateien mit gleichen Namen und unterschiedlicher Größe, Datensatz mit gleichem Schlüssel)
	\end{itemize}
	Meilensteine (16Pkt):
	\begin{itemize}
	\item Erstelle ein Replikationskonzept für diese Handelsgesellschaft (4 Punkte)
	\item Implementiere dieses Konzept für zwei Rechner (6 Punkte)	mind. 10 Datensätze pro Tabelle, mind. 10 Rechnungen
	\item Implementierung Logging (2 Punkte)
	\item Dokumentiere drei Fehler-/Problemfälle und entsprechende Lösungvorschläge (4 Punkte)
	\end{itemize}
\chapter{Designüberlegung}
\section{Allgemein}

\section{Konsistenzmodell}
	
\chapter{Arbeitsaufteilung}
	\tabulinesep = 4pt
	\begin{tabu}  {|[2pt]X[2.5,c] |[1pt] X[4,c] |[1pt]X[1.3,c]|[1pt]X[c]|[2pt]}
		\tabucline[2pt]{-}
		Name & Arbeitssegment & Time Estimated & Time Spent\\\tabucline[2pt]{-}
		
		Alexander Rieppel & Datenbankverbindung & 1h & 2h\\\tabucline[1pt]{-}
		Alexander Rieppel & Replikationsmanager & 4h & 0h\\\tabucline[1pt]{-}
		Dominik Backhausen & Verbindung der Clients & 1h & 0.5h\\\tabucline[2pt]{-}
		Dominik Backhausen & Replikationsmanager & 4h & 0h\\\tabucline[2pt]{-}
		Gesamt && 10h & 15h\\\tabucline[2pt]{-}
	\end{tabu}	
\chapter{Arbeitsdurchführung}

\chapter{Testbericht}
\section{Fehlerszenarien}
\chapter{Quellen}

\end{document}